%  To make a final copy of your thesis put a '%'
%  in front of the \includeonly command and run:
%    latex thesis
%    latex thesis
%    latex thesis
%    bibtex thesis
%    latex thesis
%    latex thesis
%
% See
%     http://www.ecn.purdue.edu/~mark/puthesis/#Options
% for documentclass options.

\documentclass[phys,dissertation]{tuthesis}

\usepackage{amsmath}
\usepackage{hyperref}

\setlength{\marginparwidth}{100pt}
\setlength{\textwidth}{410pt}


\graphicspath{{./figs/}}


%-------------------------------------------------------------------
% Title of thesis (used on cover and in abstract).
% The title shown must be the full, official title of the thesis.
% Superscripts and subscripts are not permitted in the title.
% Reference: TM2006, page 26.
% Use \title{Put Title Here} for a one-line title.
% Use \\ to separate lines in multi-line titles.
% Put % at the end of the last line of a title
% to avoid getting an extra space in the abstract.
% There are two forms of title: one line or more than one line.
% There are examples of both below.
% Only use one \title.
% CHANGE NEXT FIVE LINES.
%\title{An Example Thesis Done with LaTeX}
%\title{%
%  An Example Thesis Done with LaTeX\\
%  that has a Very Long Title%
%}
\title{
\large \bfseries Latex is smart enough to automatically\\
capitalize the title, but please\\
note the inverted pyramid\\
line break scheme}

% First author name with first name first is used for cover.
% Second author name with last name first is used for abstract.
% Your full name as it appears in the University records appears
% on the cover.
% Reference: TM2006 pages 26, 29.
% There are two forms of author, with and without initials.
% There are examples of both below.
% Only use one \author line.
% CHANGE NEXT TWO LINES.
\author{Russell Herman Conwell}{Conwell H., Russell}

% First is long title of degree (used on cover).
% Second is abbreviation for degree (used in abstract).
% Third is the month the degree was (will be) awarded (used on cover
% and in abstract).
% Last is the year the degree was (wlll be) awarded (used on cover
% and in abstract).
% The degree title for all doctoral candidates is ``Doctor of Philosophy''.
% The precise degree names for master's candidates appear in the list of
% ``Degrees Offered'' in the Graduate School bulletin.
% The date is the month and year that the degree is actually awarded.
% (If you have registered for ``degree only'', revise the thesis title
% page to reflect the new date on which the degree is to be awarded.)
% Reference: TM2006 pages 26--27, 30.
% CHANGE NEXT LINE?
\pudegree{Doctor of Philosophy}{PhD}{February}{2015}

% Major professor
% Use, for example:
%     \majorprof{Sarah Smith}
%     \majorprof{Amy A. Jones}
%     \majorprofs{Sarah Smith and Amy A. Jones}
%     \majorprofs{Sarah Smith, Amy A. Jones, and Lisa B. C. Brown}
% depending on the number of major professors you have.
% CHANGE NEXT LINE.
\committee{Hubert J. Farnsworth, Advisory Chair, Physics Department\\
   Somebody else, Math Department\\
   Somebody else, Physics Department\\
   Dr. Strangelove, External memeber, Coke-a-cola Company
}

% Campus (used only on cover)
% Use one of the following:
%     Philadelphia
% CHANGE NEXT LINE?
\campus{Philadelphia}

%
% My command definitions not specific to my thesis.
%
\input{mydefs}


% CHANGE NEXT TWO LINES?
% Let typing "\en" be exactly the same as typing "\ensuremath". 
\let\en=\ensuremath

% CHANGE NEXT TWO LINES?
% Set things up so \margins will show where the margins on the page are.
\newcommand{\margins}{\Repeat{Show where the margins for the page are.}{4}}

% CHANGE NEXT FIVE LINES?
% Define a \ve command with two arguments, so if it called with
%     \ve an
% it will expand to
%     {\en{a_1},~\en{a_2},\ \ldots,~\en{a_{n}}}
\newcommand{\ve}[2]{\en{#1_1},~\en{#1_2},\ \ldots,~\en{#1_{#2}}}


% To LaTeX only some parts of your thesis put the
% names of the parts to include here.  For example,
% \includeonly{front} would only process front.tex.
% \includeonly{front,introduction} would only process
% front.tex and introduction.tex.
% To print the final copy of your thesis put a '%'
% in front of the \includeonly command and run LaTeX
% three times to make sure that all cross-references
% are correct.  Then run BibTeX once and LaTeX twice
% more.
% CHANGE NEXT LINE?
%\includeonly{front,introduction}
%\includeonly{front}

\makeindex

\begin{document}

% Start a new volume for your thesis.
% All theses must have at least one volume.
% If your thesis has multiple volumes put another "\volume"
% command between chapters below.
\volume

% Front matter:
% ABSTRACT
% DEDICATION
% ACKNOWLEDGMENTS
% PREFACE
% TABLE OF CONTENTS
% LIST OF TABLES
% LIST OF FIGURES
% LIST OF ILLUSTRATIONS
% CHAPTER #s AND TITLES
% NOTES
% BIBLIOGRAPHY/REFERENCES CITED
% APPENDIX LETTERS AND TITLES
%  This is ``front matter'' for the thesis.
%
%  Regarding ``References'' below:
%      KEY    MEANING
%      PU     ``A Manual for the Preparation of Graduate Theses'',
%             The Graduate School, Purdue University, 1996.
%      TCMOS  The Chicago Manual of Style, Edition 14.
%      WNNCD  Webster's Ninth New Collegiate Dictionary.
%
%  Lines marked with "%%" may need to be changed.
%
\begin{copyrightpage}
   \vspace{1.8in}
   \begin{center}
      \copyright \\
      Copyright \\
      2015\\
   \vspace{0.5cm}
   by\\
   \vspace{0.5cm}
   Your Name \\
   \vspace{-8pt}
   \noindent\rule{3.5cm}{0.4pt}\\

   All Rights Reserved
   \end{center}
\end{copyrightpage}

  % Abstract is required.
  % Note that the information for the first paragraph of the output
  % doesn't need to be input here...it is put in automatically from
  % information you supplied earlier using \title, \author, \degree,
  % and \majorprof.
  % Reference: PU 17.
\begin{abstract}
Abstract text
\end{abstract}

  % Dedication page is optional.
  % A name and often a message in tribute to a person or cause.
  % References: PU 15, WNNCD 332.
\begin{dedication}
The dedication
\end{dedication}

  % Acknowledgements page is optional but most theses include
  % a brief statement of apreciation or recognition of special
  % assistance.
  % Reference: PU 16.
\begin{acknowledgments}
  This is the acknowledgments.
\end{acknowledgments}

  % The preface is optional.
  % References: PU 16, TCMOS 1.49, WNNCD 927.
\begin{preface}
  This is the preface.
\end{preface}

  % The Table of Contents is required.
  % The Table of Contents will be automatically created for you
  % using information you supply in
  %     \chapter
  %     \section
  %     \subsection
  %     \subsubsection
  % commands.
  % Reference: PU 16.
\tableofcontents

  % If your thesis has tables, a list of tables is required.
  % The List of Tables will be automatically created for you using
  % information you supply in
  %     \begin{table} ... \end{table}
  % environments.
  % Reference: PU 16.
\listoftables

  % If your thesis has figures, a list of figures is required.
  % The List of Figures will be automatically created for you using
  % information you supply in
  %     \begin{figure} ... \end{figure}
  % environments.
  % Reference: PU 16.
\listoffigures

  % List of Symbols is optional.
  % Reference: PU 17.
%\begin{symbols}
%  $m$& mass\cr
%  $v$& velocity\cr
%\end{symbols}
%
%  % List of Abbreviations is optional.
%  % Reference: PU 17.
%\begin{abbreviations}
%  abbr& abbreviation\cr
%  bcf& billion cubic feet\cr
%  BMOC& big man on campus\cr
%\end{abbreviations}
%
%  % Nomenclature is optional.
%  % Reference: PU 17.
%\begin{nomenclature}
%  Alanine& 2-Aminopropanoic acid\cr
%  Valine& 2-Amino-3-methylbutanoic acid\cr
%\end{nomenclature}
%
%  % Glossary is optional
%  % Reference: PU 17.
%\begin{glossary}
%  chick& female, usually young\cr
%  dude& male, usually young\cr
%\end{glossary}

% this adds the blank space and "Chpater"  column required by the TUHB
\chapterscolumn



\chapter{The First Chapter}

An intro.

\section{Section one}

In galaxy far far away.

\chapter{The Second Chapter}

The original was better.

\section{Herding cats}

Oh.

\subsection{Is fun}

Bender. \cite{futurama}

%\bibliographystyle{amsalpha} %The style you want to use for references.
\bibliography{template.bib} 
%The files containing all the articles and books you ever referenced.

% Appendices are optional.
% Appendices are not necessarily a part of every thesis.
% An appendix is used for supplementary illustrative material,
% original data, computer programs, and other material that
% is not necessarily appropriate for inclusion within the
% text of your thesis.
% Reference: TM2006 page 33.
% Use "\appendix" for one appendix or "\appendices" for more than one
% appendix.
% CHANGE NEXT 7 LINES?
\appendices
\chapter{Kitties}

Everyone loves kitties.




%\include{demo-citations}
%\include{demo-figures}
%\include{demo-mathematics}
%\include{demo-multicols}
%\include{demo-tables}
%\include{demo-text}


% Notes and footnotes are optional.
% Reference: TM2006 page 34.
% I have not implemented this yet.  Mark Senn 2002-06-03
%%\include{notes}

% A vita is optional for masters theses
% and required for doctoral dissertations.
% Reference: TM2006 page 13.
% CHANGE NEXT LINE?
%\include{vita}

\end{document}

% LaTeX won't read after the \end{document} command.
% You can put notes to yourself or LaTeX input not
% ready for use here if you'd like.
